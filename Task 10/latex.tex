\documentclass[12pt]{article}
\usepackage{geometry}
\usepackage{listings}
\usepackage{color}
\usepackage{graphicx}
\usepackage{hyperref}
\usepackage{fancyhdr}    % For custom headers and footers
\usepackage{geometry}    % For setting page margins
\usepackage{amsmath}     % For advanced mathematical formatting
\usepackage{hyperref}    % For hyperlinks
\usepackage{float}

% Page margin

% Header and Footer settings
\pagestyle{fancy}
\fancyhead[L]{\textbf{Assignment : 2}}         % Left header
\fancyhead[R]{\textbf{FAST NUCES}}     % Right header
\fancyfoot[C]{\thepage}                % Center footer with page number

% Title Information

\date{Oct 25,2024} % No date

% Page settings
%\geometry{
%	a4paper,
%	total={170mm,257mm},
%	left=20mm,
%	top=20mm,
%}

% Listings settings
\definecolor{codegreen}{rgb}{0,0.6,0}
\definecolor{codegray}{rgb}{0.5,0.5,0.5}
\definecolor{codepurple}{rgb}{0.58,0,0.82}
\definecolor{backcolour}{rgb}{0.95,0.95,0.92}

\lstdefinestyle{mystyle}{
	backgroundcolor=\color{backcolour},   
	commentstyle=\color{codegreen},
	keywordstyle=\color{blue},
	numberstyle=\tiny\color{codegray},
	stringstyle=\color{codepurple},
	basicstyle=\ttfamily\footnotesize,
	breakatwhitespace=false,         
	breaklines=true,                 
	captionpos=b,                    
	keepspaces=true,                 
	numbers=left,                    
	numbersep=5pt,                  
	showspaces=false,                
	showstringspaces=false,
	showtabs=false,                  
	tabsize=2
}

\lstset{style=mystyle}

\title{Task \# 10}
\author{Muhammad Shafeen\\
	FAST University Peshawar\\
	Department of Computer Science\\
	Course: Operating System\\
	Instructor: Saad Ahmad}


\begin{document}
	
	\section{Kill Command}
	This command is used to politely kill the running process , we have different sub commands or kill processes under the kill command


\begin{figure}[H]
	\centering
	\includegraphics[width=0.7\textwidth]{"../../../../../../home/shafeenyousafzai/Pictures/Screenshots/Screenshot from 2024-10-25 08-47-38"}
	\caption{Showing processes that are runnning }
	\label{fig:screenshot-from-2024-10-25-08-47-38}
\end{figure}
\begin{figure}[H]
	\centering
	\includegraphics[width=0.7\textwidth]{"../../../../../../home/shafeenyousafzai/Pictures/Screenshots/Screenshot from 2024-10-25 08-47-58"}
	\caption{killing the bash with process id {7097}}
	\label{fig:screenshot-from-2024-10-25-08-47-58}
\end{figure}
\begin{figure}[H]
	\centering
	\includegraphics[width=0.7\textwidth]{"../../../../../../home/shafeenyousafzai/Pictures/Screenshots/Screenshot from 2024-10-25 08-48-08"}
	\caption{As you can see the process we just killed has been terminated}
	\label{fig:screenshot-from-2024-10-25-08-48-08}
\end{figure}

\section{5.1.1.1 Exercise}
\subsection{Question : }
The integer representation for the SIGTERM signal		
\subsection{Answer : }
The integer representation for the SIGTERM is 15
\subsection{Question : }
The PID of your current active bash shell cess, we will use
using the ps command
\subsection{Answer : }
The integer representation for bash shell on my pc : 4482 , 4847 , 7116 , 7514
	\begin{figure}[H]
		\centering
		\includegraphics[width=0.7\textwidth]{"../../../../../../home/shafeenyousafzai/Pictures/Screenshots/Screenshot from 2024-10-25 09-03-53"}
		\caption{This show the current processes}
		\label{fig:screenshot-from-2024-10-25-09-03-53}
	\end{figure}
	\begin{figure}[H]
		\centering
		\includegraphics[width=0.7\textwidth]{"../../../../../../home/shafeenyousafzai/Pictures/Screenshots/Screenshot from 2024-10-25 09-06-33"}
		\caption{Killing the terminal process}
		\label{fig:screenshot-from-2024-10-25-09-06-33}
	\end{figure}
	\begin{figure}[H]
		\centering
		\includegraphics[width=0.7\textwidth]{"../../../../../../home/shafeenyousafzai/Pictures/Screenshots/Screenshot from 2024-10-25 09-07-00"}
		\caption{The process has been killed}
		\label{fig:screenshot-from-2024-10-25-09-07-00}
	\end{figure}
\section{Kill () command\\ C code)}	
\begin{lstlisting}[language=C, caption={Mini-Shell using execvp}]
	#include<stdio.h>
	#include<sys/types.h>
	#include<signal.h>
	#include<unistd.h>
	int main()
	{
		printf("\nMuhammad Shafeen\n");
		printf("22P-9278\n");
		printf("BAI-5A\n");
		
		int x=10;
		int y=20;
		int sum=x+y;
		printf("Sum of %d and %d is : %d\n",x,y,sum);
		
		kill(getpid(),9);
		
		printf("The program has been killed\n");
		return 0;
	}	
\end{lstlisting}		
\subsection{Screenshots of C code}

\begin{figure}[H]
	\centering
	\includegraphics[width=0.7\textwidth]{"../../../../../../home/shafeenyousafzai/Pictures/Screenshots/Screenshot from 2024-10-25 09-17-46"}
	\caption{The code for killing a process using C code}
	\label{fig:screenshot-from-2024-10-25-09-17-46}
\end{figure}
\begin{figure}[H]
	\centering
	\includegraphics[width=0.7\textwidth]{"../../../../../../home/shafeenyousafzai/Pictures/Screenshots/Screenshot from 2024-10-25 09-17-51"}
	\caption{Output of the code}
	\label{fig:screenshot-from-2024-10-25-09-17-51}
\end{figure}

\begin{figure}[H]
	\centering
	\includegraphics[width=0.7\textwidth]{"../../../../../../home/shafeenyousafzai/Pictures/Screenshots/Screenshot from 2024-10-25 09-26-06"}
	\caption{The code to use 15 as kill }
	\label{fig:screenshot-from-2024-10-25-09-26-06}
\end{figure}
\begin{figure}[H]
	\centering
	\includegraphics[width=0.7\textwidth]{"../../../../../../home/shafeenyousafzai/Pictures/Screenshots/Screenshot from 2024-10-25 09-26-11"}
	\caption{Output of the code}
	\label{fig:screenshot-from-2024-10-25-09-26-11}
\end{figure}
\section{5.1.5.1 Exercise}
\subsection{Code using fork , Child signal-ing parent to kill}
I have used the execl command to show the processes 
and then killed the process with SIGTERM and then display the processes after killing it
\begin{lstlisting}[language=C, caption={Mini-Shell using execvp}]
#include<stdio.h>
#include<sys/types.h>
#include<signal.h>
#include<unistd.h>
#include<stdlib.h>
int main()
{
	printf("\nMuhammad Shafeen\n");
	printf("22P-9278\n");
	printf("BAI-5A\n");
	
	int x=10;
	int y=20;
	int sum=x+y;
	printf("Sum of %d and %d is : %d\n",x,y,sum);
	pid_t pid;
	pid=fork();
	if(pid==0)
	{
		pid_t pid2;
		pid2=fork();
		if(pid2==0)
		{
			printf("Showing processess before killing it\n");
			execl("/bin/ps","ps -au",(char *)NULL);
			perror("execl failed");
		}
		// sleep(5);
	}
	else
	{
		kill(getppid(),15); //9
		printf("Showing processess after killing it\n");
		execl("/bin/ps","ps -au",(char *)NULL);
		perror("execl failed");
		exit(EXIT_FAILURE);
		return 0;
	}
}
\end{lstlisting}
\begin{figure}[H]
	\centering
	\includegraphics[width=0.7\textwidth]{"../../../../../../home/shafeenyousafzai/Pictures/Screenshots/Screenshot from 2024-10-25 10-00-02"}
	\caption{Code for kill() using fork()}
	\label{fig:screenshot-from-2024-10-25-10-00-02}
\end{figure}
\begin{figure}[H]
	\centering
	\includegraphics[width=0.7\textwidth]{"../../../../../../home/shafeenyousafzai/Pictures/Screenshots/Screenshot from 2024-10-25 10-00-08"}
	\caption{Output of the code for kill() and fork() )}
	\label{fig:screenshot-from-2024-10-25-10-00-08}
\end{figure}

	
\subsection{Code using fork , parent signal-ing child to kill}
I have used the execl command to show the processes 
and then killed the process with SIGTERM and then display the processes after killing it
\begin{lstlisting}[language=C, caption={Mini-Shell using execvp}]
#include<stdio.h>
#include<sys/types.h>
#include<signal.h>
#include<unistd.h>
#include<stdlib.h>
int main()
{
	printf("\nMuhammad Shafeen\n");
	printf("22P-9278\n");
	printf("BAI-5A\n");
	
	int x=10;
	int y=20;
	int sum=x+y;
	printf("Sum of %d and %d is : %d\n",x,y,sum);
	pid_t pid;
	pid=fork();
	if(pid==0)
	{
		sleep(1);
		pid_t pid2;
		pid2=fork();
		if(pid2==0)
		{
			printf("Showing processess before killing it\n");
			execl("/bin/ps","ps -au",(char *)NULL);
			perror("execl failed");
		}
		kill(getpid(),15); //9
		// sleep(5);
	}
	else
	{
		sleep(2);
		printf("Showing processess after killing it\n");
		execl("/bin/ps","ps -au",(char *)NULL);
		perror("execl failed");
		exit(EXIT_FAILURE);
		return 0;
	}
}	

\end{lstlisting}
\begin{figure}[H]
	\centering
	\includegraphics[width=0.7\textwidth]{"../../../../../../home/shafeenyousafzai/Pictures/Screenshots/Screenshot from 2024-10-25 10-08-12"}
	\caption{}
	\label{fig:screenshot-from-2024-10-25-10-08-12}
\end{figure}
\begin{figure}[H]
	\centering
	\includegraphics[width=0.7\textwidth]{"../../../../../../home/shafeenyousafzai/Pictures/Screenshots/Screenshot from 2024-10-25 10-08-18"}
	\caption{}
	\label{fig:screenshot-from-2024-10-25-10-08-18}
\end{figure}
\section{Signal Handling Exercise}
\begin{lstlisting}[language=C, caption={Mini-Shell using execvp}]
#include <signal.h>
#include <stdio.h>
#include <unistd.h>
int sigCounter = 0;
void sigHandler(int sigNum)
{
	printf("Signal received is %d\n", sigNum);
	++sigCounter;
	printf("Signals received %d\n", sigCounter); }
int main()
{
	signal(SIGINT, sigHandler);
	while(1)
	{
		printf("Hello Dears\n");
		sleep(1);
	}
	return 0;
}	
	
\end{lstlisting}
\begin{figure}[H]
	\centering
	\includegraphics[width=0.7\textwidth]{"../../../../../../home/shafeenyousafzai/Pictures/Screenshots/Screenshot from 2024-10-25 10-13-03"}
	\caption{The code for infinite loop}
	\label{fig:screenshot-from-2024-10-25-10-13-03}
\end{figure}
\begin{figure}[H]
	\centering
	\includegraphics[width=0.7\textwidth]{"../../../../../../home/shafeenyousafzai/Pictures/Screenshots/Screenshot from 2024-10-25 10-23-40"}
	\caption{Execution of infinite loop}
	\label{fig:screenshot-from-2024-10-25-10-23-40}
\end{figure}


\end{document}